\documentclass{article}
\usepackage[utf8]{inputenc}
\usepackage[margin=1in]{geometry}
\usepackage{hyperref}
\usepackage{amsmath}

\title{Math 314 (Mathematical Circles) Community Service Journal}
\author{Matthew Chin}
\date{Fall 2020}

\begin{document}
\maketitle
\tableofcontents


\newpage
\section{Introduction}
\paragraph{}This is a journal of documentation for all volunteering and community service work for MATH 314, Mathematical Circles class for the Fall 2020 semester. I will be volunteering with Eastside College Preparatory in Palo Alto, California. I am working with teacher Marianne Chowning-Dray and her Algebra 2 students in the ninth and eleventh grades.

\part{Week of October 19-23}
\section{Wednesday, 23 September 2020}
\paragraph{} This semester, my volunteer work will be with Eastside College Preparatory School in Palo Alto, California, which serves students from low-income and/or first-generation families in middle and high school who want to prepare themselves for college. For such students, they may not have the resources to prepare for post-secondary education and enter the school to be able to grasp these opportunities. Some fears I might have is with effective communication between me and the teacher(s) and students that I will be working with at first. Additionally, time management might be another issue as I must be tighter with my time in other classes and activities outside of class (i.e. organizations, work, etc.) to spend time with the students that I will be working with. 

I want to understand how the students are being taught mathematics, and figure out what they may not exactly understand as they continue their studies. I hope to bridge any gaps in understanding mathematical concepts or the "lies" about the do's and do-not's about solving problems in making mathematics easier to comprehend based on previous concepts from previous math classes taken. It is necessary for me to make learning mathematics concepts fun and in interactive ways aside from rote memorization, and find any possibilities in interactive or hands-on activities. 

\section{My Thoughts about \emph{A Mathematician's Lament} through "Mathematics and Culture", Monday, 19 October 2020}

\paragraph{}The author, Paul Lockhart, explains about education in general, all of which occur before college. It is not just focused on mathematics, but across other subjects, of the confinements by federal and state standards expecting their schools to teach certain concepts in the governments’ ways. Most of what he explains about education is true where there is no true real-world application of what is learned in a course as well as the alternative methods used to teach to students and “hope” they understand these concepts under the assumption they would get into college. 

I only disagree with Lockhart when he mentioned that “mathematics is an art” (22). I personally believe it is a science and has significant applications in STEM careers; the only reason it may be viewed as an art might be because of the language that is used that over time I am able to understand through practice, and how unique of an “art” math is as is studying music, which require some components of math in terms of time signatures, fractional notes and rests, etc. I would consider any social sciences, liberal arts, or humanities to be part of the arts; since math doesn’t fall under either one, I consider math a science due to its many applications and its impact on innovation in the said fields of study. How an equation works, how it is visually interpreted, and why use it more often than another one is what makes it applied by observing, comparing, and contrasting as would be the case in science. 

Other than that, everything else that Lockhart explains in detail is absolutely true about mathematics education and the failure in schools to actually teach mathematics. My secondary education was about playing catch-up. There were already smart students who took geometry in middle school, but at the same time all of us students were getting used to the Common Core State Standards being put in full force my freshman year. It was weird getting a Geometry workbook and going through the book realizing I had to understand the subject their way and write triangle congruency proofs in very restricted terms and abbreviations. Lockhart addressed how mathematics education has been systemically devalued in schooling and is simply something that well-off students who want to enter STEM careers need to get into top colleges. Sometimes, I think that is necessary to run with the crowd even though it is a bad but true narrative. 

While my high school stayed with the CC curriculum for all of the classes I took, I have to say that I was pretty privileged going to a public high school having taken Advanced Placement (AP) and Honors courses in math, science, and history. I did labs and Socratic seminars and was just one student of many in an uphill pipeline to get into a four-year college, “competing” with other students for good grades and several more AP classes year after year. It really wasn’t fun, but I was grateful for taking additional classes (Orchestra, Photography, Debate) to use some classroom skills in real-life and/or practical applications. Some of these classes tried to go out of the restricted norms of Common Core and state standards, but there were limits based on a list of what we were to learn and what we had to know on the day of a test. 

STEM was basically a common goal for students in my high school; we aimed high but did not settle for the minimum; therefore, it was about memorizing equations and the unit circle, then moving on to another topic. Sometimes the weighted classes I took I ended up learning a lot from since there are learning standards from the College Board mixed in with the state standards, which somewhat devalued what I’d get out of classes. There was no “my way” of actually understanding content since the learning environment was so competitive where high school for me was already college. Despite that feeling, when entering actual college, I felt more prepared having endured the stress of not actually understanding everything that I learned and spent a little more time understanding what Lockhart was hoping mathematics education is and what could be better but may never get there. Math sometimes is just memorizing and building off of equations, where teaching of its uses may not change any time soon.  

\section{Tuesday, 20 October 2020: First Interaction with Students at Eastside}

\paragraph{}Today was my first official day working with students at Eastside College Prep with teacher Marianne Chowning-Dray. I popped into breakout rooms from time to time and was able to respond to some of the students' questions with an in-class assignment known as an "exit ticket". It seemed as if these ninth grade students understood what they were learning about functions and how they work, which is an important concept in Algebra 2. This was \textbf{the last half-hour of their seventy-five-minute class}, where they worked on their ticket; today, they were learning about functions and how to understand their properties when altered and/or changed. 

In one breakout group, however, I drew out a graph of a function and the difference between stretching and shrinking a function like $f(x) = \sqrt{x}$ with some magnitude factor. I think that made it a little bit easier for the students there to understand what was being asked for, under the assumption that they had at least some prior experience either visualizing or drawing functions. I was provided access to the notes for the class where there is some coverage of sketching graphs, placing points, scalars affecting the functions, and whether any horizontal or vertical shift applies based on those changes. 

Most of the students understood changes based on transformations; I expected a lot more questions than I had thought considering that these students were freshmen in a college preparatory school and are already preparing themselves for college-level mathematics, taking more than one math course every year. Only in one of the breakout rooms I had hoped there was a bit more collaboration; all the cameras and mics were off and hoped that they would turn them on to see if they needed anything. Other than that, it was mostly smooth and am glad that I was able to meet and interact with these students today. 

\section{Thursday, 22 October 2020}

\paragraph{}I worked with students reviewing functions for freshmen Algebra 2 class. Most of the students were working individually; understandably, they might be a bit shy or afraid of unmuting themselves on Zoom. I was able to work with some of the in-class review questions and walk through clarifying when a function increased or decreased based on their physical appearance. I provided feedback on students' responses and was able to ask them why they chose one response over another. 

It was informative of me to provide some visual hints that they might find useful in more advanced mathematical subjects such as Calculus when working with functions and how they change over a certain period of time. I worked with Marianne to get students to communicate more, and opened up options to have students visualize their calculations so that they had a better understanding in their review, potentially for an upcoming exam. It was mostly a productive review day where there could be some improvements with verbal communication in a virtual classroom. 

\textbf{Time volunteered: 1 hour}

\section{Sunday, 25 October 2020}

\paragraph{}I provided feedback to Algebra 2 students' review assignments in preparation for a test on functions. I dealt with differences between submissions of four separate pictures versus one PDF file with four pages. All that was looked for was for completion of the assignment, but I did provide hints to some students who may not have understood concepts such as odd and even functions. Some of the short hints were about symmetry of functions based on the value of a function $f(x)$ where they should use that $x = -x$ in order to find if the function is odd, even, or neither. 

It was hard to provide some hints to some questions about the geometric interpretation of a vertical stretch or vertical shrink of a function, but I was able to provide some basic hints on simply looking at the algebra of the function. In terms of horizontal direction, if the scale of number was between 0 and 1 (not including 0 and 1), then a function was being stretched horizontal. If greater than 1, then a function would be getting shrunk horizontally. 

Another concept I applied with some hints given involved some degree of wishful thinking: finding a basic function based on the function given. If the function was $f(x) = \sqrt{2x + 3}$, then the basic function would be $f(x) = \sqrt{x}$, which would help with understanding the transformations on the function given. The feedback I provided to students scanning over their review assignments were hopefully about understanding the transformations of functions a little bit more and how they are affected.   

\textbf{Time volunteered: 1.5 hours}

\part{Week of October 26-30}
\section{Wednesday, 28 October 2020}

\paragraph{}I worked with eleventh-grade Algebra 2 students this day with complex numbers as part of an in-class activity. There was little participation with all but one of the breakout rooms that I did sweeps through. That one group I worked with, I was able to walk through some in-class exercises some guidance and hints for how to solve them based on their understanding of lecture material as well as asynchronous lectures. 

I responded to a student's questions about a fraction of complex numbers, bringing up the idea to make the denominator "look more simple" by multiplying by a conjugate of the denominator. This simply means to multiply by 1 and eliminate the "i's" from the denominator. There were six that Marianne (teacher) wanted to do, but due to some technical difficulties there were only three I was able to check in and work through with the students in their breakout rooms. The questions provided were mostly focused on how to divide and subtract complex number quantities, where I think getting over the hump was fully understanding doing nothing to a fraction of complex numbers by multiplying by a component of 1. 

\textbf{Time volunteered: 30 minutes}

\section{Thursday, 29 October 2020}

\paragraph{}I worked with Marianne's ninth-grade Algebra 2 students this day with complex numbers as part of an in-class activity. It appeared that this group of students was more willing to participate and interact with each other in the Zoom breakout rooms with their cameras on, and walked through the class exercises on the exit ticket to be completed. It was more interactive and I was able to communicate with the students and did occasional check-ins while spending more time with students in solving these exercises with basic concepts similar to that working with the eleventh grade students the day prior. There was more productivity with this class; the students were more open to explaining their work and sharing their screens whenever possible. 

I was able to offer hints to students about slight algebraic mistakes like subtracting quantities of equations, where one quantity minus a quantity is like adding two quantities, but the second quantity is multiplied by -1 followed by combining like terms based on addition and subtraction. Again, for this group of students, I was able to explain the concept of the conjugate when working with division of complex numbers. This simply is multiplying by 1; for example, if the function students working with was 5/(1 + 3i), the denominator is 1 + 3i. The conjugate to multiply by 1 is to multiply 5/(1 + 3i) to (1 - 3i)/(1 - 3i) to get a denominator of 1 - 9i2, or 1 + 9 = 10. The idea of simplicity appeared complicated, but actually makes the equation look a little bit simpler. 

I also visualized using some form of graph with number like i35. Since complex numbers i rotate in some form of four-cycle, by drawing a circle with four points, I encouraged the students to keep counting based on the four dots until they reached 35. In the end, 35 divided by 4 results in a remainder of 3, so the value is -i. They were able to understand what this was aside from the algebra of division modulo 4. 


\textbf{Time volunteered: 30 minutes}

\part{Week of November 2-6}
\section{Monday, November 2, 2020}
\paragraph{}I worked with the eleventh grade Algebra 2 students today who continued their work with complex numbers. They encountered quadratic equations and how to input the correct values of a, b, and c, based on an equation they are given if a function cannot be factored into smaller values if a function were to be set to 0. I walked through with some students how they were to understand the coefficients and constants of a function that are to be used to calculate the real or complex values of a function which have values x set a function and/or equal to 0. These are the "zeros" of the function that hit the x-axis or when y = 0. 

Overall, there was little to no confusion with incorporating the values of coefficients in a function to the values needed to plug in to the quadratic formula, which can be confusing at times algebraically. It can determine a lot about a function's values and whether the x values to find using the quadratic formula are either real or complex. There are always at least two values since the typical $b^2 - 4ac$ is added to or subtracted to from the "b" coefficient value. The students understood the algebra; the only problems were about understanding the coefficients of the function that can affect how a function changes or is changed. 

\textbf{Time volunteered: 30 minutes; Total hours volunteered this semester: 4.5 hours}

\section{Tuesday, November 3, 2020}

\paragraph{}I helped out with the in-class activities for the ninth grade Algebra 2 students. I was able to go between breakout rooms on Zoom with the students and asked them if they had any questions, and explain how they responded to the in-class exercises on the quadratic formula. The only confusions that some of the students that explained their answers were small algebraic mistakes; other than that, they understood what they did by explaining their work to get their solutions. 

I wanted to understand what the gap was between factoring functions and using the quadratic formula whenever needed. They eventually understood how to plug in the a, b, and c coefficients of the function into the quadratic formula whenever there was a complex number in use or if it was not solely integers that they were working with. They can check if factored properly by using the FOIL method to find the x-intercepts of a function. It might be something minor, but it is understandable to not get it the first time to apply the quadratic formula to finding the zeros of a function. I was able to provide some guidance and walk through which values of a function were to be plugged in and where in the formula may needed to be applied.

\textbf{Time volunteered: 30 minutes; Total hours volunteered this semester: 5 hours}

\section{Wednesday, November 4, 2020}

\paragraph{}I helped out the eleventh grade Algebra 2 students with understanding the visualizing of quadratic functions and how to determine if a function is negative or positive. I was able to ask the students how they came to solutions to questions with their in-class activity if they were able to get an answer. I looked through the exit ticket that the students were working on, and it looked like they had sufficient practice sketching parabolas and understanding the idea of "symmetry" in a function. When some students' breakout rooms I went through I responded to questions from them, I physically visualized the symmetry of the parabolic functions. In addition, when asking about the function's vertex and how the function shifts either left-right or up-down, they were able to understand how to calculate the values for both x-intercepts and y-intercepts. There was gradual understanding of sketching functions and interpreting how to shift them one way or another, as well as change the sign of the function if the it is either positive or negative. That helped them understand the symmetry of the functions they were working with.


\textbf{Time volunteered: 30 minutes; Total hours volunteered this semester: 6 hours}

\section{Thursday, November 5, 2020}

\paragraph{}Ninth grade Algebra 2 students today had to work on in-class activities analyzing quadratic functions by sketching them and calculating components such as the range, vertex, and y-intercept among others. I had hoped that there would be a bit more communication between students in breakout rooms, but it appeared that some problems may have been associated with understanding shifts in either the x-axis or the y-axis while determining the properties of a quadratic function. I tried to provide a few hints to students about the symmetry of quadratic equations, and how visualizing that would help in determining the vertex (where the direction of the "sign" changes), as well as determining the range and y-intercept of the function. I spent more time in some rooms than others to see if they would be able to explain their solutions to the questions given on their exit tickets. Most explanations went well, and was mostly positive if they needed to ask the "why" and "how" questions when they received their answer. 

\textbf{Time: 30 minutes; Total hours volunteered this semester: 6.5 hours}

\part{Week of November 9-13}
\section{Monday, November 9, 2020}

\paragraph{} I worked with eleventh grade students understand more about functions and how if or when one is changed in a direction or based on some sort of shift. In addition, they worked on functions required to completing the square. Already, this is hidden wishful thinking that some of these students may not have known about to make a function appear "easier" to solve, aside from multiplying by the conjugate of something else when working with complex numbers. Marianne recommended to me to pay attention to only one or two of the groups who might need some of my guidance, since it appeared that some of these groups may not be as hard-working as other groups, nor would be the case for the ninth grade students also taking Algebra 2 that I help as well. 



\textbf{Time: 30 minutes; Total hours volunteered this semester: 7 hours}

\section{Tuesday, November 10, 2020}

\paragraph{} Today, I helped the group of ninth grade students with in-class activities involving calculating the vertex of a function, as well as completing the square. Most of the time that I spent with students was walking through algebraic calculations, since it appeared that most understood the visuals of quadratic functions, and generally knowing that the vertex was when the function shifted its sign from positive to negative (a negative function, vertex is the maximum), or its sign from negative to positive (a positive function, vertex is the minimum). Most of the breakout rooms that I went through, the students were able to get through two problems: one about completing the square; the other question about sketching a parabolic function and determining its vertex. I also let them know about the symmetry of that kind of function based on the fact that at that vertex, that point signifies when the function changes sign which can define many other things later in more advanced math classes. I was able to walk through with a student in understanding what a function is to be visualized as to serve as an aide to calculating the vertex. This helped with figuring out how to use the coefficients of the function in order to solve for the point on the function that was the vertex, or the place where the function changes sign, or when it changes direction opposite of how it started. 

\textbf{Time: 45 minutes; Total hours volunteered this semester: 7.75 hours}


\section{Wednesday, November 11, 2020}

\paragraph{}The eleventh grade Algebra 2 students continued working on exercises involving parabolas and quadratic functions. In addition, they were taught about absolute value functions and how there exist two different possible "equations" based on both the positive and negative values that $x$ can be when plugged into $f(x)$. They worked more with completing the square as well as simplifying the appearance of equations to make it easier to solve for the values of that equation. This includes cancelling any factors of a least common denominator or squaring functions to get rid of radical signs on portions of equations, among other strategies. I mostly provided guidance to some breakout rooms who may have had questions with computing or combining like terms. It was straightforward where finding the absolute value meant that if ${\left\lvert 2x - 4 \right\rvert} = 9$, then that means that $x$ can be found using two values by calculating: $2x - 4 = -9$ or $2x - 4 = 9$, which would lead to two different variables for $x$. 

\textbf{Time: 30 minutes; Total hours volunteered this semester: 8.25 hours}


\section{Thursday, November 12, 2020}

\paragraph{}Today, I spent time with the ninth grade Algebra 2 students as they worked on solving for values in absolute value equations, radical expressions, and other rational equations. The only concerns were making equations look simpler by getting rid of a radical sign: squaring an entire equation if a square root was found, or cubing an entire equation given that there is a cubed-root radical sign on one side of the equation. These students were able to catch on and realize of their mistakes in computing any values that would land on that function. Although I had continued to help students after class finished, they were willing to understand what they may have missed in their lectures, as well as fixed their own miscalculations. 

These students were hard-working in getting to understand some of the mistakes they make, as well as the most ideal way to solve their in-class exercises. This may not be a Math Circle, but there are some fundamentals in Algebra 2 that are also from my MATH 314 course that are key in more advanced math courses. Functions that the students work with do involve some degrees of symmetry as well as wishful thinking, such as rewriting equations to remove radical signs, completing the square, or taking the conjugate of a complex number, among other forms of "wishful thinking". The most basic of calculations are necessary to understand Algebra 2, and these students are eager to understand these concepts considering their backgrounds that they may not be as privileged as other high school or college-preparatory students as they are. Some of the breakout rooms are willing to ask questions and do collaborate with each other. Verbal communication appears to be missing in mathematics classes even during a pandemic as another way of understanding material they are given in the virtual classroom. 

\textbf{Time: 1 hour, Total hours volunteered this semester: 9.25 hours}

\part{Week of November 16-19}

\section{Monday, November 16, 2020}

\paragraph{}Today, I helped out the eleventh grade students with problems about inequalities and testing for the values. They could be either true or false when determining if a number used is less than or greater than what is the expected value, among other inequalities that can be solved. I noticed how in one of the breakout rooms I went to that all but one camera was on, and only one of the students had questions and needed guidance on the in-class exit ticket exercises regarding inequalities. Only until after I helped out solving one of the questions did other students begin to ask questions and unmute themselves. Sometimes, it takes a little bit of pushing or guidance to get started, and that is understandable if students are being "thrown" at with concepts and questions to do on their own and understand with concerns about how to apply this into their daily life. 

On the other hand, another breakout room I entered the students there were communicating to one another, and were actually excited that I came in. At most, there was one question from one of the students but they were just happy that I was there because the last time I helped them out we were pretty productive in understanding some concepts with finding values associated with functions. 

\textbf{Time: 30 minutes; Total hours volunteered this semester: 9.75 hours}


\section{Tuesday, November 17, 2020}

\paragraph{}Today I helped work on the in-class exercises with the ninth grade students with the same questions on inequalities and working on interval notation. Since there was more time today to work with the students in the breakout rooms, I spent some time asking students in different rooms if they understood how they reached their solutions, as well as follow-up questions that were also addressed to other students in their respective "rooms". Only one of the rooms did I have to offer some assistance with number lines in determining if certain values in a function met the requirements for inequalities. 

\textbf{Time: 45 minutes; Total hours volunteered this semester: 10.5 hours}

\section{Wednesday, November 18, 2020}

\paragraph{}Today was a review day for the eleventh grade Algebra 2 students. They worked on a review sheet for their topics including quadratic equations, inequalities, complex numbers, and completing the square. They also covered solving equations with hidden ideas of wishful thinking: keeping it simple, reducing equations, and making a problem easier to solve. They also had another activity to work on prior to the review sheet. I asked the students in their breakout rooms if they wanted to share some of their answers and how they calculated to solve their problems. There were some errors that one group of students had trying to complete the square of an equation, that I realized that they were able to fix on their own and understand the other components and properties of that function. It was good that they were able to catch on with their errors. 

\textbf{Time: 30 minutes; Total hours volunteered this semester: 11 hours}

\section{Analysis of \textit{A Mathematician's Lament} and Questions with Marianne, Instructor (due 11/19/20)}

\paragraph{} 
Some of the material mentioned in Lockhart's writing towards the end of the book also made some sense of what exactly adding and multiplying are for my Modern Algebra class. These are just some ways of figuring out patterns, and we create others to figure out something else about numbers. His statement that mathematics is being taught in a "structural environment" is very concerning considering how coursework is planned in the classroom as a routine and a requirement to be able to graduate from high school or college. The notion of "odd" and "even" numbers and how to situate them as such was beauty in the sense that putting odd numbers together as dots placing odd-numbered dots in formation of an L resulted in a square. 

Additionally, some concepts of group theory were made simpler of Lockhart's explanations: putting dots together showed equality and symmetry with an even number, but not with an odd number of dots. The sum of two even numbers were even, sum of two odd numbers resulted in an even sum, and an odd number of dots and an even number of dots resulted in an odd sum of dots. If it were not for visual and hands-on application, I may not have understood mathematics the right way: with different ways of expressing understanding, sufficient geometric/visual interpretation, and a little bit of wishful thinking (if not, a lot of what could have been instead of what is given). 

The past Thursday after class Thursday, November 12, 2020, with the ninth grade Algebra 2 students, I spent some time asking a few questions about how Marianne taught mathematics to her students, even during a pandemic. For the record, she has read \textit{A Mathematician's Lament} and knew of the concerns with mathematics education about how students are to understand the subject in limited perspectives, instead of having them find their own methods of problem solving. Eastside stresses mathematics in their students' curriculum, given that they take two math classes per year each of their four years of high school. Considering the personal situations they may be in, they are very lucky to be able to do so, considering at other high schools students may not be able to take a total of eight math courses in four years. Algebra 2 is considered a key part of mathematics since working with functions are necessary to understand other subjects such as physics, computer science, economics, statistics, etc. It prepares them for college-level mathematics. 

According to Marianne, the teacher that I have been working with, mathematics should be taught as an artistic expression. Unfortunately, due to COVID and significant changes in mathematics education, students are put at different levels where some students get ahead, some are behind, and others play catch-up. Since there appears to be inequality in these opportunities, mathematics is more of a school routine and less of an art. Marianne thinks that mathematics is "fitness": training for future careers that no longer feels artistic. Students are able to find their different ways of thinking and how to deal with new forms of communication and logic. She does think that by exercising the mind, her students will be able to pick up new ideas and even pick up on their own mistakes. These skills will motivate them to do better in the future. 

These college preparatory students are preparing to go to college, very likely in STEM careers dependent on mathematics, and it appears that Eastside sticks to that. These students are getting prepared for the SAT and other standardized tests needed to get in despite their economic or academic barriers. At the least, Geometry is the last math class "required" to take the SAT, but the school also puts the need for Algebra 2 as well because students work with functions and some trigonometry. Because of this, she believes that by exercising the brain, there will be some room made for how students are to think differently when studying mathematics. However, that was not necessarily the case with the students I worked with. 

Lockhart discusses of the flaws of teaching Geometry in the classroom: definitions, postulates, wordiness of terms, and structured proofs involving specific terms on similarities, properties, and well-known triangular congruence abbreviations like SSS, AAS, ASA, CPCTC, etc. Marianne calls on the need for geometry to fully return to the algebraic and visual observations made in the subject. Without the visualization, there would be a lack of creative and wishful thinking necessary in geometry, considering that the logic behind proofs is "archaic" and "disappointing". Some schools teach Geometry after Algebra 1, while other schools opt for Algebra 2 after Algebra 1. Algebra 2 focuses completely on functions, with little Trigonometry at the end. Pre-calculus comes afterwards with even more Trigonometry, and so the inconsistencies in education early on in high school make a difference in success in higher education. 

Due to the pandemic, it is considerably hard to apply real hands-on activities such as volume and integration in calculus, graphing exponential growth in Algebra 2, and the overall collaboration of students with their algebraic skills. While some of these projects appear to be interesting, to students, since they are not in math circles, they only want to succeed and are less interested overall in problem solving in their math classes. She also noticed some flaws in project-based learning, including the learning barriers prior to creating projects, the rigor that comes with the projects, and whether or not students feel a sense of accomplishment after completing their project, exercise, or examination. 

There is also some concern brought about by the hierarchy of how students take math classes in middle school and high school. For those who want to play catch-up, some school districts make it harder for them to advance one math class, while for others who are advanced, they go to a community college to take college-level math courses if their high school does not offer calculus or statistics. Since some schools either lack funding, teachers, or space to offer students more advanced math, Marianne suggests that there should be more opportunities opened to students to explore them. 

Mathematics classes should be broadened in depth, ways of thinking, and diverse methods of understanding concepts, potentially changing the learning standards. Additionally, math circles should be encouraged more outside of school to supplement and broaden their education on the subject. It should not just be limited to those who can afford these programs, nor should it be only given to students who can either afford it or are more academically merited than other students. It is best to spread different ways of problem solving to any students who want to know more than what is in their textbooks or on their learning outcomes. Both her and I agree that students may need to do more than what is required of them for classes and that teachers should provide more support for learning about broadening problem solving skills as Lockhart would suggest to his students.

\section{Thursday, November 19, 2020}

\paragraph{}Today I worked with the ninth grade students on their review about inequalities, quadratic equations and the quadratic formula, solving values in equations, and figuring out anything about complex numbers. I was excited to hear that some of the students understood my way of visualizing imaginary number $i$ and that it is like a 4-cycle per the jargon of Abstract/Modern Algebra. For me personally, visualizing the cycle of $i, i^2 = -1, i^3 = -i, i^4 = 1$ also aides with algebraically finding the remainder of a quotient or calculating a number modulo 4. 

I checked in the breakout rooms if they needed assistance with understanding parabolic/quadratic equations. The only things they needed help with were visualizing if there was to be a minimum or maximum of a function based on the coefficient "$a$" for $f(x) = ax^2 + bx + c$. If $a<0$, function would face down and have a maximum at the vertex; if otherwise, the function would face up and have a minimum at the vertex. These were more understandable interpretations that the students were able to adapt to. 

\textbf{Time volunteered: 45 minutes, Total hours volunteered this semester: 11.75 hours}

\section{Sunday, November 22, 2020}

\paragraph{}
I spent the day providing feedback to both of the ninth and eleventh grade Algebra 2 students on their review sheets regarding quadratic equations, inequalities, and complex numbers. I noticed that many of the students had some problems understanding what to do with the intersection of two absolute value inequalities. There were four cases that they had to check if they were true; if not, they were graphed on a number line as false. Additionally, there were some issues with plotting number line dots with either an open (not inclusive value) or a closed (inclusive value), where some students did not pay attention to such a key direction that they could have spent more time with. 

Another thing that I noticed as I was providing feedback was the fact that for some responses which required the quadratic formula, there were several "no solution" responses. It was not only of what they learned of the quadratic formula that some students missed, hoping they got real rational numbers, but missed real irrational numbers as well as complex and imaginary numbers. I remember working on activities with students involving the quadratic equation and imaginary numbers. I was hoping they were not confined to getting integers, but for some situations, that is what I saw. I also noticed, on the other hand, that most of the students understood questions that are introduced in introductory physics in terms of the maximum or minimum - the vertex of a function. 

I enjoyed volunteering with these students and hoped that there was a different way of understanding mathematics than by what was taught in the virtual classroom. It was hard because this is the first full period of time of virtual learning, so it is understandable that students may not be able to grasp onto everything they learn since they are learning from home. Little did I know about how important it is for these students to understand Algebra 2 as it will prepare them for college-level mathematics with functions, complex numbers, and science classes such as physics. I never thought that I needed Algebra 2 until taking this Math Circles class, and it made me understand how the little things start from the math classes early on in high school. I believe these students that I worked with will do great things and are very hardworking considering that the freshmen I worked are taking two math classes their first year and will continue to do so throughout high school. This would truly prepare them for college even though there could be some improvements to learning mathematics in the class room for future students. They will have a good future and would be prepared for college-leveled classes.  

\textbf{Time volunteered: 1.5 hours, Total hours volunteered this semester: 13.25 hours}


\part{Summary of Volunteer Work}

\paragraph{}Volunteering with Marianne Chowning-Dray and her ninth and eleventh grade Algebra 2 students at Eastside College Preparatory completely convinced me how students who want to enter STEM careers need to learn math early and from the get-go. Trying to play catch-up with the other "brighter" students in middle school and high school will always be unsuccessful unless they find more work in mathematical problem solving outside of the classroom. If that is the case for students, then there should be more opportunities for them to do so. This includes after-school programs or more inclusive programs for any students of any backgrounds to take part in mathematical circles and other math activities. This would help broaden a student's math curriculum given that there are limits and learning outcomes in an average classroom. 

In total, I volunteered \textbf{13.25 hours this semester}. I learned a lot about how some students cannot take their education for granted considering how many of the students that I work with may be from low-income neighborhoods and/or may be first generation students eager to enter a four-year college or university. While mathematics may be an exercise for some students who may want to enter STEM careers, it is necessary to fully understand its functions and purposes instead of memorizing equations, formulas, or basic concepts about certain disciplines of math. Hopefully, there are more opportunities to explore different ways and means of problem solving that are available to students to expand or broaden a student's mathematical education, and one where some work can be done outside of the classroom and can have fun learning and truly applying their skills. 
\end{document}
