\documentclass{article}
\usepackage[utf8]{inputenc}
\usepackage{amsmath}

\title{Personal Demo}
\author{Matthew Chin}
\date{January 23, 2020}

\begin{document}

\maketitle

\section{Introduction}

Hi there, how are you?

\section{Creating a Table}

Below is an example of creating a table.

\begin{table}[h]
\centering
    \begin{tabular}{|l|l|l|}
    \hline
    Column 1 & Column 2 & Column 3\\
    \hline
    Row 1a & Row 1b & Row 1c\\
    Row 2a & Row 2b & Row 2c\\
    Row 3a & Row 3b & Row 3c\\
    \hline
    \end{tabular}
    \caption{The basic code used to create a table.}
\end{table}

\section{Making Equations}

\noindent
The mass-energy equivalence is described by the famous equation 

\[E=mc^2\]
\noindent
discovered in 1905 by Albert Einstein.
In natural units ($c$ = 1), the formula expresses the identity

\begin{equation}
    E=m.
\end{equation}{}

\noindent
Another equation:

\begin{equation} \label{eq1}
\begin{split}
A &= \frac{\pi r^2}{2} \\
 &= \frac{1}{2} \pi r^2.
\end{split}
\end{equation}

\section{Additional Equations}

Here is another set of equations that are written:

\begin{equation} \label{eu_eqn}
    e^{\pi i} + 1 = 0
\end{equation}
\noindent
The beautiful equation \ref{eu_eqn} is known as the Euler equation.



\noindent
This is an example of a long equation that uses the multi-line environment:

\begin{multline*}
    p(x) = 3x^6 + 14x^5y + 590x^4y^2 + 19x^3y^3
    - 12x^2y^4 - 12xy^5 + 2y^6 - a^3b^3
\end{multline*}

\noindent
Here is an aligned equation written out and formatted:
\begin{align*}
    (a + b)^2 + (a - b)^2 &= a^2 + 2ab + b^2 + (a - b)^2 \\
    &= a^2 + 2ab + b^2 + a^2 - 2ab + b^2 \\
    &= 2ab^2 + 2b^2.
\end{align*}

\end{document}
